\chapter{はじめに}
\label{section01}

%%%%
\section{研究背景と目的}

\section{論文構成}



%\chapter{序論}
%\label{section01}
%
%%%%%
%\section{本研究の背景}
%現在, 日本では化石燃料中心の産業や社会構造を太陽光などのクリーンエネルギー中心の産業や社会構造に転換していく取り組みであるGX(グリーン・トランスフォーメーション)の実現が目指されている. そのため, 太陽光発電設備や風力発電設備などの再生可能エネルギー機器が分散型電源として電力系統に投入され, その数は段々と増加している. しかしこのような分散型電源は多種多様であり, 日射量や風速の違いなど電源の種類によっては出力の安定性が懸念されている. これら不安定な電源の投入, 拡大によって電力系統の需給バランスの調整は今まで以上に難しくなると予想される. このような背景から2021年度に需給バランスの調整のための電力を調整力として取引する電力需給調整市場という市場が開設された.~\cite{{jukyuu}}
%%%%%%参考文献  
%\par
%電力需給調整市場では時間スケールの長い経済負荷配分制御(EDC : Economic load Dispatching Control)に相当する三次調整力や時間スケールの短いガバナフリー制御に相当する一次調整力などのメニューがあり, それら調整力の広域調達がすでに始まっている. また, それらの調整力を組み合わせた複合約定というメニューも存在する. その中で蓄電池を用いた蓄電設備は太陽光や風力などの再生可能エネルギーを用いた発電設備と比較すると安定的な電源としての運用が可能であり, 調整力の安定的な電源としての利用が期待されている.~\cite{{sice24amano}, {ieej24tamura}}
%また, 文献~\cite{{ieej25okada}}
%では, 電力市場を考慮した計画(入札)に関する研究が報告されている. 一方で, 三次調整力および一次調整力の複合約定を想定した, 実時間における制御方策については検討, 検証がされていない.
%
%\section{研究の目的と概要}
%本論文では, 複数の蓄電池によって構成された蓄電ステーションにおいて三次調整力および一次調整力の複合約定を可能とする分散型の運用方策を提案し, 提案した運用方策を用いた実機実験を行い, 運用方策の有効性を検証する. また, 全ての実機実験においてシミュレーションとの比較も行う. \par
%以下に本論文の概要を示す.\\
%2章では本論文で対象としている電力需給調整市場の説明, 想定している蓄電ステーションの構成システムなどについて述べる. 3章では各調整力用モードでの分散型の出力目標値決定方法, 各モードの切り換えアルゴリズムについて述べる. 4章では提案した運用方策についての実機実験を行った結果とそれに伴った有効性の確認について述べる. 5章では本論文の統括をおこなう.
%

