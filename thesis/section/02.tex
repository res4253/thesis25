\chapter{電力市場と想定のシステムについて}
\label{cec_market}
%%%%%あとから考える
本章では, 対象としている電力需給調整市場についてと, その中での各調整力の定義や入札方法などのシステムの説明, また本論文で想定している蓄電ステーションのシステムについても説明する. 
\section{電力需給調整市場}\label{sec_market1}
過去, 一般送配電事業者は電力供給区域の周波数制御, 需給バランス調整のために自社の発電機を余分に動かすことによって必要な調整用の電力(調整力)を確保していた. しかし, このような余分な電力の発電をすることは非効率であるため, 2017年度より一般送配電事業者が確保する調整力は原則として公募で調達することとなった. その後, 多くの電源等への参加機会の公平性確保, 調達コストの透明性および適切性の確保, 調整力の効率的な確保の観点から2021年度に電力需給調整市場が創設された. この電力需給調整市場の取引商品としては一次調整力, 二次調整力\raise0.2ex\hbox{\textcircled{\scriptsize{1}}}, 二次調整力\raise0.2ex\hbox{\textcircled{\scriptsize{2}}}, 三次調整力\raise0.2ex\hbox{\textcircled{\scriptsize{1}}}, 三次調整力\raise0.2ex\hbox{\textcircled{\scriptsize{2}}}がある. 各商品の要件, 概要をFig.~\ref{fig_syouhin}に示す. 

\vspace{-0.5cm}

\begin{figure}[h]
\centering
\includegraphics[width=0.8\textwidth]{fig/syouhin.pdf}
%%%%キャプション考える
\caption{各調整力の要件, 概要}
\label{fig_syouhin}
\end{figure}


なお, 本論文で扱う一次調整力は商品概要において一次調整力, 三次調整力は商品概要において三次調整力\raise0.2ex\hbox{\textcircled{\scriptsize{1}}}を想定している. 一次調整力とは応動時間が10秒以内の時間スケールの短い調整力であり, 三次調整力は応動時間が15分以内と時間スケールの長い調整力となっている. 
この一次調整力と三次調整力は実際にFig.~\ref{kaisetu}に示すように2024年度以降ではどちらも広域調達が開始している. \par
 
また, 各調整力の必要量のピークとなる地点が必ずしも同時に発生するものではないという不等時性を踏まえ, さらなる調整力確保の効率化の観点から, 1つのリソースが同時に複数商品に入札することが可能である場合, その入札方法を許容する複合約定というものも存在する. 
%%%%商品概要電力調整力取引所のスライドから引用
\begin{figure}[h]
\centering
\includegraphics[scale=0.3]{fig/kaisetu.pdf}
\caption{各調整力市場の状況}
\label{kaisetu}
\end{figure}

\section{調整力の入札のシステム}
本節では対象としている一次調整力や三次調整力\raise0.2ex\hbox{\textcircled{\scriptsize{1}}}などの週間商品である調整力の入札システムについて説明する. \par
電力需給調整力取引所は取引対象の週の土曜日から金曜日までの1週間分の必要な調整力を予想し, 取引会員との入札規模を決定, 取引対象の前週の月曜日の14時に取引会員に公表される. 取引会員は火曜日の14時までに

自身の設備の出力量を考慮し, 各調整力の入札時間単位ごとに入札の応募をする. 電力需給調整力取引所は各取引会員の入札の応募に対して調整力の出力可能量と金額から必要量分の調整力の約定を15時までにおこなう.  取引対象当日に約定した調整力の中で実際に必要になった調整力は約定した取引会員に出力の要請をして取引会員の設備から調整力が出力される. もし約定していた出力量が出力できない場合や応動時間以内に出力できない場合は罰金などの措置が取られる. そのため約定した取引会員は自身の設備が揚力発電であれば約定量の発電ができるほどの揚水, 蓄電ステーションであれば約定量分の充電量など約定量分の電力を確保しておく必要がある. 






\section{想定する蓄電ステーション}
三次調整力は\ref{sec_market1}節でも説明した通り時間スケールの長い調整力であり, 水力や揚力発電といった発電システムによる電源でも対応できるが一次調整力のような時間スケールの短い調整力には対応することが難しい. しかし蓄電池は時定数が短く一次調整力のような時間スケールの短い調整力にも対応することができ, 天候などによる出力変化も少ないため安定した電源である. そのため本論文では蓄電池を用いた蓄電ステーションを想定し一次調整力と三次調整力の複合約定を可能とする.\par
想定する蓄電ステーションのシステムをFig.~\ref{fig_bat}に示す. 蓄電ステーションは$n$台の蓄電池が配置されており, 各蓄電池は蓄電池用PCS(Power Conditioning System)を通して電力系統に接続されている. 蓄電池から出力された電力は各PCSで交流電力に変換され電力系統に流れる. この蓄電ステーションで達成すべき運用目標は電力系統との接続点での出力$P$(=各PCSから出力される電力の合計出力値)を基本的には三次調整力指令値$P^{\mathrm{I3}}$に一致させ, 周波数変動があった場合には一次調整力として必要に応じて出力$P$を調整することである. 各PCSは出力目標値$P^{\mathrm{r}}_{i}$, 
$i \in N =\{1,2,\ldots,n\}$を自身で決定しながらも, 運用目標の達成を可能とする分散型の出力目標値決定方策を本論文では考える. 三次調整力での出力目標値を$P_i^{\mathrm{r3}}$, 一次調整力での出力目標値を$P_i^{\mathrm{r1}}$とすると各PCSが自身で決定する出力目標値$P_i^{\mathrm{r}}$は
\begin{align}
P^{\mathrm{r}}_{i} = 
 P^{\mathrm{r3}}_{i} + P^{\mathrm{r1}}_{i}
\end{align}
として決定する.

\begin{figure}[h]
\centering
\includegraphics[width=0.9\textwidth]{fig/batst.eps}
\caption{Battery Storage Station}
\label{fig_bat}
\end{figure}

