
%% 作成する論文にあわせて doctor, master, bachelor を指定してください.
%% draft モードは, 図の入れこみを省略し, タイプセット時間を省略します.
%\documentclass[doctor,draft]{dsr-thesis}
\documentclass[bachelor]{dsr-thesis}

%% 必要に応じてパッケージを追加してください.
%\usepackage[dvips]{graphicx,color}
% \usepackage[dvipdfmx]{color}
\usepackage[dvipdfmx]{graphicx,color}
\usepackage[cmex10]{amsmath}
\usepackage{amsthm}
\usepackage{amssymb}
\usepackage{ascmac}
\usepackage{here}
\usepackage{listings}
\usepackage{jlisting}
\usepackage{url}
\usepackage{colortbl}
\usepackage{float}
\usepackage{here}
\usepackage{udline}
\usepackage{comment}
%%%%% subfigure
%\usepackage{subfigure}
\usepackage{subcaption}
%%%%% subfloat
%\usepackage{subfig}
%%%%% subcaption
% \usepackage[hang,small,bf]{caption}
% \usepackage[subrefformat=parens]{subcaption}
% \captionsetup{compatibility=false}
\captionsetup[subfigure]{labelformat=simple}
\renewcommand{\thesubfigure}{(\alph{subfigure})}

%%%%% for jlisting
\renewcommand{\lstlistingname}{List}
%%%
\lstset{
   breaklines=true,
   basicstyle=\ttfamily\small,
   keywordstyle={\bfseries \color[cmyk]{0,1,0,0}},
   commentstyle={\ttfamily \color[cmyk]{1,0.4,1,0}},
   stringstyle={\ttfamily \color[rgb]{0,0,1}},
   frame=tb,
   framesep=1zw,
   numbers=left,
   numberstyle=\tiny,
   stepnumber=1,
   numbersep=10pt,
   backgroundcolor=\color{white},
   tabsize=2,
   showspaces=false,
   showstringspaces=false,
   columns=[1]{fullflexible},
   lineskip=-0.5zw}
%%%
\lstdefinestyle{sh}{
   language=sh,
   morekeywords={elif}}
%%%
\lstdefinestyle{csh}{
   language=csh,
   deletekeywords={printf}}
%%%
\lstdefinestyle{matlab}{
   language=matlab,
   morekeywords={mkdir, strcat, saveas, imread, im2frame, movie2avi}}
%%%
\lstdefinestyle{pde}{
   language=c++,
   morekeywords={byte, boolean}}
\lstdefinestyle{c}{
   language=C,
   morekeywords={byte, boolean}}

%subfigureの並べた図一つ一つのキャプションの位置を変更
\def\subfigcapskip{-4mm}
%subfigre全体のキャプションの位置を変更
\def\subfigbottomskip{0pt}

%% ドキュメント情報

\title{
\ul{切り換えシステムを用いた糖代謝数理モデルによる}\\
\ul{状態推定器の構築}
}



\etitle{
Title in English
}
\author{田辺 裕翔}
\eauthor{TANABE yuto}
\school{富山大学}
\eschool{University of Toyama}
\enteryear{4}
\defensejyear{7}
\major{工学部 工学科 電気電子工学コース}
\emajor{Major in English}
\identify{St. Number}
\adviser{平田 研二 教授}
% \subadviser{氏名} % 主査が二人の場合
\symbol{logo_y03.eps}
\date{令和 8 年  月  日}
\edate{February 1, 2020.} % 英語概要用の日付



\makeatletter
  \renewcommand{\section}{%
    \@startsection{section}{1}{\z@}%
    {0.3ex plus -.5ex minus -.2ex}{.3ex plus .2ex}%
    {\Large\bf}}
\renewcommand{\subsection}{%
    \@startsection{subsection}{2}{\z@}%
    {0.3ex plus -.5ex minus -.2ex}{.3ex plus .2ex}%
    {\large\bf}}
  \renewcommand{\subsubsection}{%
    \@startsection{subsubsection}{3}{\z@}%
    {0ex plus -.5ex minus -.2ex}{.3ex plus .2ex}%
    {\bf}}
    %{\normalfont\large\headfont\raggedright}}
\makeatother


%%%%%%%%%%%%%%%%%%%%%%%%%%%%%%%%%%%%%%%%%%%%%%%%%%%%%%%%%%%%
%%%%%%%%%%%%%%%%%%%%%%%%%%%%%%%%%%%%%%%%%%%%%%%%%%%%%%%%%%%%
\begin{document}
\maketitle
%\clearpage

%%%%%%%%%%%%%%%%%%%%%%%%%%%%%%%%%%%%%%%%%%%%%%%%%%%%%%%%%%%%
%% 英語概要
%\begin{eabstract}
 %\input{abstract/enabst.tex}
%\end{eabstract}

%%%%%%%%%%%%%%%%%%%%%%%%%%%%%%%%%%%%%%%%%%%%%%%%%%%%%%%%%%%%
%% 日本語概要
%\begin{abstract}
% \input{abstract/jaabst.tex}
%\end{abstract}


%%%%%%%%%%%%%%%%%%%%%%%%%%%%%%%%%%%%%%%%%%%%%%%%%%%%%%%%%%%%
%%%%%%%%%%%%%%%%%%%%%%%%%%%%%%%%%%%%%%%%%%%%%%%%%%%%%%%%%%%%
%\cleardoublepage

%%%%%%%%%%%%%%%%%%%%%%%%%%%%%%%%%%%%%%%%%%%%%%%%%%%%%%%%%%%%
%% 目次など
\frontmatter
\tableofcontents
\listoffigures
\listoftables
\cleardoublepage
\mainmatter
\pagenumbering{arabic}
\setcounter{page}{1}
\renewcommand{\thefootnote}{\fnsymbol{footnote}}

%%%%%%%%%%%%%%%%%%%%%%%%%%%%%%%%%%%%%%%%%%%%%%%%%%%%%%%%%%%%
%% 本論
%\input{chapter/00list.tex}
%%%%%%%%%% Introduction
\cleardoublepage
%\input{section/01.tex}
%\chapter{電力市場と想定のシステムについて}
\label{cec_market}
%%%%%あとから考える
本章では, 対象としている電力需給調整市場についてと, その中での各調整力の定義や入札方法などのシステムの説明, また本論文で想定している蓄電ステーションのシステムについても説明する. 
\section{電力需給調整市場}\label{sec_market1}
過去, 一般送配電事業者は電力供給区域の周波数制御, 需給バランス調整のために自社の発電機を余分に動かすことによって必要な調整用の電力(調整力)を確保していた. しかし, このような余分な電力の発電をすることは非効率であるため, 2017年度より一般送配電事業者が確保する調整力は原則として公募で調達することとなった. その後, 多くの電源等への参加機会の公平性確保, 調達コストの透明性および適切性の確保, 調整力の効率的な確保の観点から2021年度に電力需給調整市場が創設された. この電力需給調整市場の取引商品としては一次調整力, 二次調整力\raise0.2ex\hbox{\textcircled{\scriptsize{1}}}, 二次調整力\raise0.2ex\hbox{\textcircled{\scriptsize{2}}}, 三次調整力\raise0.2ex\hbox{\textcircled{\scriptsize{1}}}, 三次調整力\raise0.2ex\hbox{\textcircled{\scriptsize{2}}}がある. 各商品の要件, 概要をFig.~\ref{fig_syouhin}に示す. 

\vspace{-0.5cm}

\begin{figure}[h]
\centering
\includegraphics[width=0.8\textwidth]{fig/syouhin.pdf}
%%%%キャプション考える
\caption{各調整力の要件, 概要}
\label{fig_syouhin}
\end{figure}


なお, 本論文で扱う一次調整力は商品概要において一次調整力, 三次調整力は商品概要において三次調整力\raise0.2ex\hbox{\textcircled{\scriptsize{1}}}を想定している. 一次調整力とは応動時間が10秒以内の時間スケールの短い調整力であり, 三次調整力は応動時間が15分以内と時間スケールの長い調整力となっている. 
この一次調整力と三次調整力は実際にFig.~\ref{kaisetu}に示すように2024年度以降ではどちらも広域調達が開始している. \par
 
また, 各調整力の必要量のピークとなる地点が必ずしも同時に発生するものではないという不等時性を踏まえ, さらなる調整力確保の効率化の観点から, 1つのリソースが同時に複数商品に入札することが可能である場合, その入札方法を許容する複合約定というものも存在する. 
%%%%商品概要電力調整力取引所のスライドから引用
\begin{figure}[h]
\centering
\includegraphics[scale=0.3]{fig/kaisetu.pdf}
\caption{各調整力市場の状況}
\label{kaisetu}
\end{figure}

\section{調整力の入札のシステム}
本節では対象としている一次調整力や三次調整力\raise0.2ex\hbox{\textcircled{\scriptsize{1}}}などの週間商品である調整力の入札システムについて説明する. \par
電力需給調整力取引所は取引対象の週の土曜日から金曜日までの1週間分の必要な調整力を予想し, 取引会員との入札規模を決定, 取引対象の前週の月曜日の14時に取引会員に公表される. 取引会員は火曜日の14時までに

自身の設備の出力量を考慮し, 各調整力の入札時間単位ごとに入札の応募をする. 電力需給調整力取引所は各取引会員の入札の応募に対して調整力の出力可能量と金額から必要量分の調整力の約定を15時までにおこなう.  取引対象当日に約定した調整力の中で実際に必要になった調整力は約定した取引会員に出力の要請をして取引会員の設備から調整力が出力される. もし約定していた出力量が出力できない場合や応動時間以内に出力できない場合は罰金などの措置が取られる. そのため約定した取引会員は自身の設備が揚力発電であれば約定量の発電ができるほどの揚水, 蓄電ステーションであれば約定量分の充電量など約定量分の電力を確保しておく必要がある. 






\section{想定する蓄電ステーション}
三次調整力は\ref{sec_market1}節でも説明した通り時間スケールの長い調整力であり, 水力や揚力発電といった発電システムによる電源でも対応できるが一次調整力のような時間スケールの短い調整力には対応することが難しい. しかし蓄電池は時定数が短く一次調整力のような時間スケールの短い調整力にも対応することができ, 天候などによる出力変化も少ないため安定した電源である. そのため本論文では蓄電池を用いた蓄電ステーションを想定し一次調整力と三次調整力の複合約定を可能とする.\par
想定する蓄電ステーションのシステムをFig.~\ref{fig_bat}に示す. 蓄電ステーションは$n$台の蓄電池が配置されており, 各蓄電池は蓄電池用PCS(Power Conditioning System)を通して電力系統に接続されている. 蓄電池から出力された電力は各PCSで交流電力に変換され電力系統に流れる. この蓄電ステーションで達成すべき運用目標は電力系統との接続点での出力$P$(=各PCSから出力される電力の合計出力値)を基本的には三次調整力指令値$P^{\mathrm{I3}}$に一致させ, 周波数変動があった場合には一次調整力として必要に応じて出力$P$を調整することである. 各PCSは出力目標値$P^{\mathrm{r}}_{i}$, 
$i \in N =\{1,2,\ldots,n\}$を自身で決定しながらも, 運用目標の達成を可能とする分散型の出力目標値決定方策を本論文では考える. 三次調整力での出力目標値を$P_i^{\mathrm{r3}}$, 一次調整力での出力目標値を$P_i^{\mathrm{r1}}$とすると各PCSが自身で決定する出力目標値$P_i^{\mathrm{r}}$は
\begin{align}
P^{\mathrm{r}}_{i} = 
 P^{\mathrm{r3}}_{i} + P^{\mathrm{r1}}_{i}
\end{align}
として決定する.

\begin{figure}[h]
\centering
\includegraphics[width=0.9\textwidth]{fig/batst.eps}
\caption{Battery Storage Station}
\label{fig_bat}
\end{figure}


%\chapter{分散型制御方策}
\label{cec_strategy}


本章では, 提案する分散型の制御方策において各調整力へ対応するモードと各モード切り換えのアルゴリズムについて説明する.

\section{三次調整力モード}\label{sec_long}
本論文%本論文
では三次調整力モードとして価格提示による分散最適化~\cite{{iscie24akutsu}}
を用いた制御系によって三次調整力に対応する. 本節ではその分散最適化を用いた制御方策について説明する. 
%%%%%%
\subsection{蓄電ステーションの集中最適化問題}\label{sec_st_1}
%%%%%
本論文%本論文
で想定している蓄電ステーションにはPCSのほかに, システム全体の管理者であるユーティリティーが存在するとする. \par
そのユーティリティーは蓄電ステーション全体としての損失の最小化と運用目標である蓄電ステーション全体としての出力$P$(=各PCSの出力合計$\sum_{i=1}^nP_i$)と三次調整力指令値$P^{\mathrm{I3}}$の一致を達成するため, 以下の集中最適化問題を解き, 各PCSの出力目標値$P^{\mathrm{r3}}_{i}$を決定する. 


%集中最適化の式
\begin{subequations} \label{eq_all}
\begin{align}
 \min_{
  P^{\mathrm{r3}}_{i}
  }
  \quad
  &
  \sum_{i=1}^n
  w_{i}
  (P^{\mathrm{r3}}_{i})^{2}
  \label{eq_all1}
  \\
  %%%%%
  \mathrm{subject \ to} 
  \quad
  &
  -P_i^{\mathrm{lim}} \leq 
  P_i^{\mathrm{r3}} \leq
  P_i^{\mathrm{lim}}
  \label{eq_all2}
  \\
  &
  P-P^{\mathrm{I}3}=0
  \label{eq_all3}
\end{align}
\end{subequations}

ここで, $w_i$は各PCSの出力調整のための重み係数であり, (\ref{eq_all1})式は各PCSの出力の合計の最小化を表している. 不等式制約(\ref{eq_all2})式の$P_i^{\mathrm{lim}}$は各PCSの定格容量を表し, 等式制約(\ref{eq_all3})式は 全体としての出力が三次調整力指令値$P^{\mathrm{I}3}$に一致させることを表している. (\ref{eq_all})式の最適解を$(P_i^{\mathrm{r}3})^*$と表す.\par
これはユーティリティーが集中最適化問題(\ref{eq_all})式を解き, その最適解$(P_i^{\mathrm{r}3})^*$を各PCSへ提示するような中央集中型の運用である. しかしこのような中央集中型の運用は今後の蓄電拠点の需要増加に伴う拠点の拡大などによりユーティリティーの計算量が増大し対応することが困難となる可能性が高い. %この部分用相談
そのため, 各PCSが自身の出力目標値を決定する分散型の運用を考える.

\subsection{単純に分散化した最適化問題}
(\ref{eq_all})式のユーティリティーが解くべき最適化問題を単純に分散化し, 各PCSが解くとすると以下の式のようになる.

%単純に分散化した式
\begin{subequations} \label{eq_dis}
\begin{align}
 \min_{
  P^{\mathrm{r3}}_{i}
  }
  \quad
  &
  w_{i}
  (P^{\mathrm{r3}}_{i})^{2}
  \label{eq_dis1}
  \\
  %%%%%
  \mathrm{subject \ to} 
  \quad
  &
  -P_i^{\mathrm{lim}} \leq 
  P_i^{\mathrm{r3}} \leq
  P_i^{\mathrm{lim}}
  \label{eq_dis2}
\end{align}
\end{subequations}

(\ref{eq_dis})式の最適化問題の解を$(P^{\mathrm{r3}}_i)^\#$であらわす. (\ref{eq_dis})式は各PCSが自身の出力目標値を最適化することを意味しており, 全体としての最適化となっていない. また, 
%以下考え中
(\ref{eq_all3})式での等式制約による全体としての出力値を三次調整力指令値に合わせていたことができないため, $(P^{\mathrm{r3}}_{i})^\#$が$(P_i^{\mathrm{r}3})^*$と一致することは期待できない.\par
%%%%%
\subsection{価格提示による分散最適化}
%%%%%
そこで, ユーティリティーが各PCSに対して付加的な価格信号$p_i$を提示し, 各PCSは提示された価格を考慮した自身の出力目標値の最適化問題を解くことでその最適解が$(P_i^{\mathrm{r}3})^*$と一致するように誘導することを考える. 各PCSが新たに解くユーティリティーから提示される価格$p_i$を含む1次項を追加した最適化問題は
%価格含んだ分散最適化問題
\begin{subequations} \label{eq_dis_p}
 \begin{align}
  \min_{
  P^{\mathrm{r3}}_{i}
  }
  \quad
  &
  w_{i}
  (P^{\mathrm{r3}}_{i})^{2}
  +
  p_{i} P^{\mathrm{r3}}_{i}
  \label{eq_dis_p1}
  \\
  %%%%%
  \mathrm{subject \ to} 
  \quad
  &
  -P_i^{\mathrm{lim}} \leq 
  P_i^{\mathrm{r3}} \leq
  P_i^{\mathrm{lim}}
  \label{eq_dis_p2}
 \end{align}
\end{subequations}

となり, この最適化問題を解くことで各PCSは自身の出力目標値を決定する. (\ref{eq_dis_p})の最適解を

$(P_i^{\mathrm{r}3})^{\flat}(p_i)$であらわす.\par

\subsection{KKT条件の比較}
この分散最適化を達成するためにはユーティリティーは$(P_i^{\mathrm{r}3})^{\flat}(p_i)=(P_i^{\mathrm{r}3})^*$となるような適切な価格信号を提示しなければならない. そこで, もとの集中最適化問題(\ref{eq_all})式と新たな分散最適化問題(\ref{eq_dis_p})式のKarush-Kuhn-Tucker(KKT)条件を比較する.




\begin{subequations} \label{eq_allkkt}
 % \setlength{\abovedisplayskip}{0pt} % 上部のマージン
 % \setlength{\belowdisplayskip}{0pt} % 下部のマージン
 \begin{align}
  \setlength{\parskip}{0cm}
  \setlength{\itemsep}{0cm}
  %%%%%%%%%% 
  2w_i(P_i^\mathrm{r3})
  +\lambda-\mu_i^1+\mu_i^2
  =0
  % \label{eq:s03-facility_opt-min}
  \\
  %%%%%%%%%%
-P_i^{\mathrm{lim}}-P_i^\mathrm{r3} \le 0,
 \quad
 \mu_i^1(-P_i^{\mathrm{lim}}-P_i^\mathrm{r3})=0,
 \quad
 \mu_i^1\ge0
  % \label{eq:s02-facility_opt-lneq} 
  \\
  %%%%%%%%%% 
P_i^\mathrm{r3}-P_i^{\mathrm{lim}} \le 0,
 \quad
 \mu_i^2(P_i^\mathrm{r3}-P_i^{\mathrm{lim}})=0,
 \quad
 \mu_i^2\ge0
 %\label{eq:s03-facility_opt-eq}
 \\
  %%%%%%%%%% 
 i=1,\dots,n
 \nonumber \\
  %%%%%%%%%%
 P-P^{\mathrm{I}3}=0
 \end{align}
\end{subequations}
%%%%%%%%%% equation
%%%%%%%%%% facility_KKT
%%%%%%%%%% Decentralized with Price_KKT
%%%%%%%%%% equation
\begin{subequations} \label{eq_dis_pkkt}
 % \setlength{\abovedisplayskip}{0pt} % 上部のマージン
 % \setlength{\belowdisplayskip}{0pt} % 下部のマージン
 \begin{align}
  \setlength{\parskip}{0cm}
  \setlength{\itemsep}{0cm}
  %%%%%%%%%% 
  2w_i(P_i^{\mathrm{r3}})
  +p_i-\mu_i^1+\mu_i^2
  =0
  % \label{eq:s03-facility_opt-min}
  \\
  %%%%%%%%%%
 -P_i^{\mathrm{lim}}-P_i^\mathrm{r3} \le 0,
 \quad
 \mu_i^1(-P_i^{\mathrm{lim}}-P_i^\mathrm{r3})=0,
 \quad
 \mu_i^1\ge0
  \\
  %%%%%%%%%% 
P_i^\mathrm{r}-P_i^{\mathrm{lim}} \le 0,
 \quad
 \mu_i^2(P_i^\mathrm{r3}-P_i^{\mathrm{lim}})=0,
 \quad
 \mu_i^2\ge0
 \end{align}
\end{subequations}

(\ref{eq_allkkt})式が(\ref{eq_all})式のKKT条件であり, (\ref{eq_dis_pkkt})式が(\ref{eq_dis_p})式のKKT条件である. ここで, $\mu_i^1, \mu_i^2$は不等式制約(\ref{eq_all2})式及び(\ref{eq_dis_p2})式におけるLagrange乗数であり, $\lambda$は等式制約(\ref{eq_all3})式におけるLagrange乗数である.\par
2つのKKT条件の比較より,
%%%p=lambdaのしき
\begin{align}
p_i=\lambda
\label{eq_plam}
\end{align}
として価格提示をすることで$(P_i^{\mathrm{r}3})^{\flat}(p_i)=(P_i^{\mathrm{r}3})^*$を達成することができることがわかる. \par


%%%この説明が必要なければ消す
\subsection{実時間価格更新則}
本節ではユーティリティーが適切な価格$p_i=\lambda$を提示するために$\lambda$の実時間更新則を導出する. まず, 集中最適化問題(\ref{eq_all})式の双対問題を考える. 不等式制約(\ref{eq_all2})式を$h_i(P^{\mathrm{r3}}_i)$であらわすと,
%%%maxminの式文字変更要確認
\begin{align}
\max_\lambda 
\min_{\substack{P_i^\mathrm{r3} \\ 
h_i(P_i^\mathrm{r3}) \\
i=1,\dots,n}}
\sum_{i=1}^{n}
 w_{i}
  \left(
  P^{\mathrm{r3}}_{i}
  \right)^{2}
+\lambda 
\left(
P-P^{\mathrm{I}3}
\right)  
\label{eq_maxmin}
\end{align}
ここで最適解$(P_i^{\mathrm{r}3})^{\flat}$は各PCSによって決定されるとすると
%%%maxに変換文字変更要確認
\begin{align}
\max_\lambda 
 \quad
\sum_{i=1}^{n}
 w_{i}
  \left(
  (P_i^\mathrm{r3})^\flat
    \right)^{2}
+\lambda 
\left(
 P-P^{\mathrm{I3}}
\right)  
\label{eq_max}
\end{align}
%%%
となり, $\lambda$に対する最大化問題となる. この最大化問題に勾配法を適用することで$\lambda$の実時間更新則
%%%更新則
\begin{align}
\dot{\lambda}(t)=\epsilon(P(t)-P^{\mathrm{I3}}(t)),\quad \epsilon>0
\label{eq_renewprice}
\end{align}
を得ることができる. なお, $\epsilon>0$は本論文%%%本論文
で構成したフィードバック制御系の帯域幅を決定するパラメータであり, 適切な値を設定することで定常状態での$P=P^{\mathrm{I3}}$が達成できる.\par
この更新則に従いユーティリティーが価格信号を提示し, 各PCSが(\ref{eq_dis_p})を解き自身の出力目標値を決定することで$(P_i^{\mathrm{r}3})^{\flat}(p_i)=(P_i^{\mathrm{r}3})^*$となる.
%%%
%%%
%%%
\subsection{離散時間実装}
前節までで説明した制御系を実装に伴いサンプリング時間
%%%長いサンプル時間今後短いサンプル時間が出るにあたり変えたほうがいいかも
$t_s$
%%%
の離散時間系で
%%%
表現すると
%%%
各PCSが解く分散最適化問題, ユーティリティーが行う価格更新式はそれぞれ(\ref{eq_dis_p_risann}), (\ref{eq_renewprice_risann})となる.
%%%
%%%
%%%離散化した更新則
\begin{subequations} \label{eq_dis_p_risann}
 \begin{align}
  \min_{
  P^{\mathrm{r3}}_{i}[k]
  }
  \quad
  &
  w_{i}
  (P^{\mathrm{r3}}_{i}[k])^{2}
  +
  p_{i}[k] P^{\mathrm{r3}}_{i}[k]
  \label{eq_dis_p_risann1}
  \\
  %%%%%
  \mathrm{subject \ to} 
  \quad
  &
  -P_i^{\mathrm{lim}} \leq 
  P_i^{\mathrm{r3}}[k] \leq
  P_i^{\mathrm{lim}}
  \label{eq_dis_p_risann2}
 \end{align}
\end{subequations}
%%%%
%%%%
%%%%%
\begin{subequations}\label{eq_renewprice_risann}
 \begin{align}
  p_{i}[k] &= \lambda[k] \\
  \lambda[k+1] &= \lambda[k] + 
  \epsilon
  \left(
  P[k] - P^{\mathrm{I3}}[k]
  \right)
 \end{align}
\end{subequations}
%%%
%%%
三次調整力モードでは(\ref{eq_dis_p_risann})式, (\ref{eq_renewprice_risann})式に従って三次調整力に対応する. 
全体としてはFig.~\ref{fig_3jiji}に示すフィードバック制御系となり, ユーティリティーはステーションの出力$P[k]$と三次調整力指令値$P^{\mathrm{I3}}[k]$から(\ref{eq_renewprice_risann})式で価格信号$p_{i}[k]$を更新, 各PCSへ送信する. 各PCSは送信された価格$p_{i}[k]$を考慮して(\ref{eq_dis_p_risann})式に従い, 自身の出力目標値を決定, 出力する. この制御系により分散的に三次調整力に対応する. 
\begin{figure}[h]
\centering
\includegraphics[width=0.9\textwidth]{fig/3ji.pdf}
\caption{Closed-loop system}
\label{fig_3jiji}
\end{figure}




また, 三次調整力モード中は一次調整力を無視するため, 一次調整力用の出力目標値$P_i^{\mathrm{r1}}[k]$は以下の式となる. 
\begin{align}
P_i^{\mathrm{r1}}[k]=0
\end{align}

%%%
%%%
%%%
%%%一次調整力モードの話
\section{一次調整力モード}\label{sec_fast}
一次調整力モードでは, 数秒スケールでの周波数変動に対応した出力調整をすることが目的となる. 本論文では電力需給調整市場の規定~\cite{{jukyuu2}}
に従い, 各PCSが各自で周波数変動を観測し, 自身の出力目標値を決定することを想定している. \\
一次調整力モードで対応する基準周波数との偏差に対する各PCSの一次調整力用の出力目標値$P_i^{\mathrm{r1}}[k]$は
\begin{align}
 P^{\mathrm{r1}}_{i}[k] = 
 -\frac{P^{\mathrm{lim}}_{i} \times dF[k]}{F^{\mathrm{r}} \times R} 
\label{eq_dfP}
\end{align}
に従って決定する. %%%
ここで, $F^{\mathrm{r}}~[\mathrm{Hz}]$ は基準周波数, 
$dF~[\mathrm{Hz}]$は基準周波数との偏差, 
$R$ は速度調定率である.
%%%
%%%
速度調停率とは周波数変化の割合と出力の割合を比で表したもので, (\ref{eq_dfP})式は$-F^{\mathrm{r}} \times R$ の周波数変動が起こった際に, 
$P^{\mathrm{lim}}_{i}$~[kW] の放電となるような関係となっている. (\ref{eq_dfP})を図で表すと(Fig.~\ref{fig_sprate})のようになる. 
なお実際の運用において, $\pm 0.2$~Hz の周波数変動に対して応答すること, 
出力を調定率直線の $\pm 10\%$ 以内 (Fig.~\ref{fig_sprate}中の赤紫色の範囲) 
とすること, 
調定率 $R$ は 0.05 (5\%) 以下に設定することが求められる. 
%%%
\begin{figure}[h]
 \centering
 \includegraphics[width=.48\textwidth]{fig/speedrate.eps}
\caption{Speed-rate of each PCS}
 \label{fig_sprate}
\end{figure}

また, 一次調整力モード中は時間スケールの短い周波数変動に対応するため, ユーティリティーは価格更新を行わない. よって価格の更新則は以下のようになる. 
\begin{subequations}\label{eq_renewprice_risann_fast}
 \begin{align}
  p_{i}[k] &= \lambda[k] \\
  \lambda[k+1] &= \lambda[k]
 \end{align}
\end{subequations}

\section{モード切り換えのアルゴリズム}
本論文%%%本論文
では時間スケールが長い三次調整力には\ref{sec_long}節の三次調整力モードで対応し, 時間スケールの短い一次調整力には\ref{sec_fast}節の一次調整力モードで対応することを考えている. つまり, 蓄電ステーション全体としての出力$P[k]$が三次調整力指令値$P^{\mathrm{I3}}[k]$に追従するまでは三次調整力モードとして動作させ, 追従したのち一次調整力モードとなり周波数変動に対応, その後また三次調整力指令値$P_i^{\mathrm{r3}}[k]$が変化すると三次調整力モードになり, 三次調整力に対応する. という制御が理想となる. その制御実現のために本論文%本論文%
では, Algorithm~\ref{alg_mode}によりユーティリティー側でのモード切り換えを行う. モードの切り換えの信号は価格信号と同様に各PCSに送信され, その信号により各PCSもモードを切り換える. 
%%%%%
蓄電ステーション全体としての出力$P[k]$が三次調整力指令値$P^{\mathrm{I3}}[k]$に追従したと判断する許容誤差を$\delta$, 
モードの変数を$m[k]$とし, $m[k]=3$は三次調整力モードを, $m[k]=1$は一次調整力モードを表している. 

%%%%%%%%%%%%%%%%%%%%アルゴリズム(原稿)
\begin{algorithm}
 \caption{mode decision}
 \label{alg_mode}
 \begin{algorithmic}[1]
  %%%%%
  \If{$m[k]=3$}
  %%%%%%%%
  \If{$|P[k] - P^{\mathrm{I3}}[k]| < \delta$}
  \State $m[k+1] = 1$
  \Else
  \State $m[k+1] = 3$
  \EndIf
  %%%%%%%%
  \Else
  %%%%%%%%
  \If{$P^{\mathrm{I3}}[k] \neq P^{\mathrm{I3}}[k-1]$}
  \State $m[k+1] = 3$
  \Else
  \State $m[k+1] = 1$
  \EndIf
  %%%%%%%%
  \EndIf
  %%%%%
 \end{algorithmic}
\end{algorithm}
%%%%%%%%%%%%%%%%%%%%

また, 三次調整力モードでは(\ref{eq_renewprice_risann})式, 一次調整力モードでは(\ref{eq_renewprice_risann_fast})式によって価格を更新する. そのため制御系全体としての価格更新の式は%%%%すごい複雑になるが一時三次合わせた各PCSがとく問題も入れるべきかもしれない
%%%
%%%%全体としての価格更新則
\begin{subequations}\label{eq_priceall}
 \begin{align}
  p_{i}[k] &= \lambda[k] \\
  \lambda[k+1] &= \lambda[k] + \Delta \lambda[k] \\
  \Delta \lambda[k] &=
  \begin{cases}
   0 & \text{if $m[k] = 1$} \\
   \epsilon \left(
   \displaystyle P[k] - P^{\mathrm{I3}}[k]
   \right) & \text{if $m[k] = 3$} 
  \end{cases}
 \end{align}
\end{subequations}
となる.
以上より, 一次および三次調整力モードの切り換えを考慮したフィードバック制御系をFig~\ref{fig_feedback}に示す. 

\begin{figure}[h]
 \centering
 \includegraphics[scale=0.7]{fig/closedloop.eps}
\caption{Closed-loop system for the composite contract of Frequency Containment and Replacement Reserve.}
 \label{fig_feedback}
\end{figure}

ユーティリティーはシステム全体での出力$P[k]$を観測, 三次調整力指令値$P^{\mathrm{I3}}[k]$と比較し, モード$m[k]$の決定と価格の更新, 各PCSへの信号の送信をおこなう. 各PCSは受信したモード$m[k]$と価格$p_i[k]$を元に自身の出力目標値$P_i^{\mathrm{r}}$を計算し, 出力する. このフィードバック制御系により, 本論文%%%本論文
で提案する. 短い時間スケールの一次調整力と長い時間スケールの三次調整力の複合約定を可能とする分散型の制御が可能となる.







%%%%%%%%%%%%%%%%%%%%%%%%%%%%%%%%%%%%%%%%%%%%%%%%%%%%%%%%%%%%
%% 付録
\appendix
%\input{huroku.tex}
%%%%%%%%%%%%%%%%%%%%%%%%%%%%%%%%%%%%%%%%%%%%%%%%%%%%%%%%%%%%
%%%%%%%%%%%%%%%%%%%%%%%%%%%%%%%%%%%%%%%%%%%%%%%%%%%%%%%%%%%%
\backmatter

%%%%%%%%%%%%%%%%%%%%%%%%%%%%%%%%%%%%%%%%%%%%%%%%%%%%%%%%%%%%
%% 謝辞
%\input{section/acknowledgement.tex}


%%%%%%%%%%%%%%%%%%%%%%%%%%%%%%%%%%%%%%%%%%%%%%%%%%%%%%%%%%%%
%% 参考文献
\setcounter{chapter}{0}%
\markright{参考文献}%
\chapter*{参考文献}
\addcontentsline{toc}{chapter}{参考文献}
\bibliographystyle{junsrt}
\bibliography{biblio}
%%%%%%
\begin{thebibliography}{10}
 \setlength{\parskip}{0cm}
 \setlength{\itemsep}{0cm}
 
 \bibitem{jukyuu}
一般社団法人 電力需給調整力取引所:需給調整市場の概要・商品要件, 
\url{https://www.eprx.or.jp/outline/docs/gaiyoushouhin_ver.4_20240401.pdf} 


\bibitem{sice24amano}
天野:
電力系統における需給調整・制御の概要, 
計測と制御, 
\textbf{63}-1, 9/13~(2024)

\bibitem{ieej24tamura}
田村:
電力系統の定置用蓄電池の役割が期待される EV,
電気学会論文誌~B, 
\textbf{144}-3, 208/211~(2024)


\bibitem{ieej25okada}
岡田ほか:
容量市場・スポット市場・需給調整市場を考慮した発電事業者の電源開発計画の評価手法,
電気学会論文誌~B, 
\textbf{145}-1, 32/45~(2025)


\bibitem{iscie24akutsu}
阿久津ほか: 
発電・蓄電・需要機器を有する拠点群により構成される仮想発電所の階層分散型運用と実験検証,
システム制御情報学会論文誌, 
\textbf{37}-9, 247/256~(2024)


\bibitem{jukyuu2}
一般社団法人 電力需給調整力取引所:取引ガイド(全商品) (第6版) 
\url{https://www.eprx.or.jp/outline/docs/guide_250314.pdf} 

\bibitem{hokkaidou}
電力広域的運営推進機関: 平成30年北海道胆振東部地震に伴う大規模停電に関する検証委員会最終報告
\url{https://www.occto.or.jp/iinkai/hokkaido_kensho/hokkaidokensho_saishuhoukoku.html} 





%\bibitem{tuchida19}
%T.~Hatanaka, Y.~Wasa, and K.~Uchida, Eds., \emph{Economically-enabled Energy
 % Management --Interplay between control engineering and economics}.\hskip 1em
 % plus 0.5em minus 0.4em\relax Springer Nature, 2019.

%\bibitem{tsuzuki20}
%T.~Suzuki, S.~Inagaki, Y.~Susuki, and A.~T. Tran, Eds., \emph{Design and
%  Analysis of Distributed Energy Management Systems --Integration of EMS, EV,
 % and ICT--}.\hskip 1em plus 0.5em minus 0.4em\relax Springer Nature, 2020.

%\bibitem{timura19}
%井村, 原 (編), 次世代電力システム設計論
 % --再生可能エネルギーを活かす予測と制御の調和--.\hskip 1em plus 0.5em minus
%  0.4em\relax オーム社, 2019.

 %%%%%
% \bibitem{sice0112}
%	 特集 グリーンイノベーションと制御理論,
%	 計測と制御,
%	 {\bf 51}-1~(2012)
%	 %%%%%
%\bibitem{sice0114}
%	 特集 大規模エネルギーマネージメントシステムを支える省エネソリュー
%	 ション,
%	 計測と制御,
%	 {\bf 53}-1~(2014)
	 %%%%%
% \bibitem{ieejb13omine}
%	 大嶺ほか, 
%	 ヒートポンプ式給湯器と電力貯蔵装置を用いた太陽光発電余剰電力利用のための需要地系統運用手法, 
%	 電気学会論文誌 B, 
%	 {\bf 133}-7, 631/641~(2013)
% \bibitem{ieejb16mori}
%	 森川, 蜷川, 
%	 ビル用マルチ空調設備の電力抑制と室温維持を時系列調整する高速リアルタイム電力料金最適制御, 
%	 電気学会論文誌 B, 
%	 {\bf 136}-11, 817/823~(2016)
 %\bibitem{KatAgu11}
%	 F. Katiraei and J. R. Ag{\"u}ero, 
%	 Solar {PV} Integration Challenges, 
%	 {\slshape IEEE Power and Energy Magazine}, 
%	 {\bf 9}-3, 62/71~(2011)
% \bibitem{ieej09hayashi}
%	 林, 
%	 分散型電源の導入拡大に対応した配電系統電圧制御の動向と展望, 
%	 電気学会論文誌B, 
%	 {\bf 129}-4, 491/494~(2009)
% \bibitem{ieej19magazine3}
%	 特集 バーチャルパワープラント構築普及に向けた最新動向,
%	 電気学会誌, 
%	 {\bf 139}-3~(2019)
	 

% \bibitem{iscie17b}
%	 阿久津ほか, 
%	 出力抑制指令への対応を可能とする蓄電池併設型太陽光発電システムにおけるインバータ群の分散制御, 
%	 システム制御情報学会論文誌, 
%	 {\bf 30}-11, 439/448~(2017)

% \bibitem{sice18}
%	 笠輪ほか, 
%	 価格提示を利用した蓄電拠点の分散制御におけるワインドアップに関
%	 する考察, 
%	 計測自動制御学会論文集, 
%	 {\bf 54}-2, 167/174~(2018)

% \bibitem{mscs19fuji}
%	 藤澤ほか, 
%	 実時間価格提示方策を利用した離散値出力型機器を含む電力需要群の
%	 分散型運用方策, 
%	 第 6 回 制御部門マルチシンポジウム, 1F2-1~(2019)

% \bibitem{mscs20aku}
%	 阿久津ほか, 
%	 実時間価格提示方策を利用した仮想発電所の分散型運用に関する考察, 
%	 第 7 回 制御部門マルチシンポジウム~(2020)
 
% \bibitem{ieeeps18koraki}
%	 D. Koraki and K. Strunz, 
%	 Wind and Solar Power Integration in Electricity Markets and Distribution Networks Through Service-Centric Virtual Power Plants, 
%	 {\slshape IEEE Transactions on Power Systems}, 
%	 {\bf 33}-4, 473/485~(2018) 

% \bibitem{ieeeps13yang}
%	 H. Yang {\sl et al.},
%	 Distributed Optimal Dispatch of Virtual Power Plant via Limited Communication,
%	 {\slshape IEEE Transactions on Power Systems},
%	 {\bf 28}-3, 3511/3512~(2013)

% \bibitem{acc14-1925}
%	 K.~Hirata,
%	 J.~P.~Hespanha and K.~Uchida: 
%	 Real-time Pricing Leading to Optimal Operation under
%	 Distributed Decision Makings, 
%	 {\slshape Proceedings of the 2014 American Control Conference},
%	 1925/1932~(2014)

% \bibitem{sice18}
%	 笠輪, 
%	 阿久津, 
%	 平田, 
%	 \newblock
%	 価格提示を利用した蓄電拠点の分散制御におけるワインドアップに関
%	 する考察, 
%	 \newblock
%	 計測自動制御学会論文集, 
%	 {\bf 54}-2, 
%	 167/174~(2018)
	 
% \bibitem{mscs20aku}
%	 阿久津ほか, 
%	 実時間価格提示方策を利用した仮想発電所の分散型運用に関する考察, 
%	 第 7 回 制御部門マルチシンポジウム~(2020)
	 

% \bibitem{iscie15}
%	 平田, 上地, 
%	 電力需要・供給バランスを実現する車載型蓄電池の分散型充電管理方
%	 策に関する考察,
%	 システム制御情報学会論文誌,
%	 {\bf 28}-11, 427/434~(2015)


 % \bibitem{ieee-sg15}
 % 	 A.~Mnatsakanyan and S.~W.~Kennedy,
 % 	 A Novel Demand Response Model with an Application for a Virtual Power Plant,
 % 	 {\sl IEEE Transactions on Smart Grid}, 
 % 	 Vol.~6, No.~1, pp.~230--237~(2015)
 % \bibitem{IEEEE_CM_JAN13}
 % 	 L.~Hern\'{a}ndez {\sl et al.},
 % 	 A Multi-Agent System Architecture for Smart Grid Management and Forecasting of Energy Demand in Virtual Power Plants,
 % 	 {\sl IEEE Communications Magazine}, 
 % 	 Vol.~51, No.~1, pp.~106--113~(2013)
 % \bibitem{ieej174}
 % 	 八太,
 % 	 需給運用・周波数調整への影響緩和技術,
 % 	 電気学会誌, 
 % 	 Vol.~137, No.~4, pp.~224--227~(2016)
\end{thebibliography}

% \nocitebib{*}
% \bibliographystylebib{junsrt}
% \bibliographybib{biblio}

%%%%%%%%%%%%%%%%%%%%%%%%%%%%%%%%%%%%%%%%%%%%%%%%%%%%%%%%%%%%
%% 業績
% %%% ALL
% \nocitepub{*}
% \bibliographystylepub{junsrt}
% \bibliographypub{alllist}
% \setcounter{chapter}{0}%
% \markright{本研究に関する論文, 講演および賞罰}%
% \chapter*{本研究に関する論文, 講演および賞罰}
% \addcontentsline{toc}{chapter}{本研究に関する論文, 講演および賞罰}
% \section*{学会論文誌}
% %%% Transactions
% \nocitetrn{*}
% \bibliographystyletrn{junsrt}
% %\bibliographystyletrn{jplain}
% \bibliographytrn{tran}
% \section*{国際学会発表}
% %%% Conferences
% \nociteintcon{*}
% \bibliographystyleintcon{junsrt}
% %\bibliographystyleintcon{jplain}
% \bibliographyintcon{intconf}
% \section*{国内学会発表}
% %%% Conferences
% \nocitedomcon{*}
% \bibliographystyledomcon{junsrt}
% %\bibliographystyledomcon{jplain}
% \bibliographydomcon{domconf}
% \section*{特許}
% %%% Patent
% \input{patent.tex}
% \section*{賞罰}
% %%% Reward and punishment
% \input{reward_and_punishment.tex}

\end{document}