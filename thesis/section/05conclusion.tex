\chapter{結論}
\label{cec_conclude}
本論文では, 複数の蓄電池とPCSで構成される蓄電ステーションにおいて, 一次調整力および三次調整力の複合約定を達成できる分散制御方策について検討した. その中で
, 早い応答が求められる一次調整力には速度調停率によって出力を決定するモードで対応し, 時間スケールの長い三次調整力には価格提示による分散最適化制御によって全体としての損失が少ない出力の決定をおこなうモードで対応し, それらのモードを自動的かつ適切に切り換えることで複合約定を達成する方策を提案した. また, 提案した方策の有効性について様々な一次調整力で対応する周波数変動のパターンを用意し, MATLAB/Simulinkを用いたシミュレーション, スケールダウンモデルの実機器を用いたで実験の結果を比較し検証した. そしてどの実験でもシミュレーションと実機実験の結果はほぼ一致し, 三次調整力モードと一次調整力モードが適切に切り換わり出力調整をおこなっていることが確認できた. そのため提案した方策で三次調整力と一次調整力の複合約定が可能であることを確認した.\par
今後の展望としては, 実際の運用を考えた各蓄電池のSOC(State of Chage), 劣化などを考慮した運用を組み込むことなどが挙げられる.
